 % !Mode:: "TeX:UTF-8"
 \input{header}
 
 
 \title{}
 \author{}
 \date{}
 
 \begin{document}
%生成首页
\maketitle
%目录
\tableofcontents
\newpage

\section{高精}
\subsection{两个非负整数相加}

\begin{lstlisting}[language={[ANSI]C},numbers=left]

//加法
string add(string a,string b)//只限两个非负整数相加
{
    const int L=1e5;
    string ans;
    int na[L]={0},nb[L]={0};
    int la=a.size(),lb=b.size();
    for(int i=0;i<la;i++) na[la-1-i]=a[i]-'0';
    for(int i=0;i<lb;i++) nb[lb-1-i]=b[i]-'0';
    int lmax=la>lb?la:lb;
    for(int i=0;i<lmax;i++) na[i]+=nb[i],na[i+1]+=na[i]/10,na[i]%=10;
    if(na[lmax]) lmax++;
    for(int i=lmax-1;i>=0;i--) ans+=na[i]+'0';
    return ans;
}
\end{lstlisting}

\subsection{大的非负整数减小的非负整数}


\begin{lstlisting}[language={[ANSI]C},numbers=left]

//减法
string sub(string a,string b)//只限大的非负整数减小的非负整数
{
    const int L=1e5;
    string ans;
    int na[L]={0},nb[L]={0};
    int la=a.size(),lb=b.size();
    for(int i=0;i<la;i++) na[la-1-i]=a[i]-'0';
    for(int i=0;i<lb;i++) nb[lb-1-i]=b[i]-'0';
    int lmax=la>lb?la:lb;
    for(int i=0;i<lmax;i++)
    {
        na[i]-=nb[i];
        if(na[i]<0) na[i]+=10,na[i+1]--;
    }
    while(!na[--lmax]&&lmax>0)  ;lmax++;
    for(int i=lmax-1;i>=0;i--) ans+=na[i]+'0';
    return ans;
}
\end{lstlisting}

\subsection{高精乘高精}


\begin{lstlisting}[language={[ANSI]C},numbers=left]

string mul(string a,string b)//高精度乘法a,b,均为非负整数
{
    const int L=1e5;
    string s;
    int na[L],nb[L],nc[L],La=a.size(),Lb=b.size();//na存储被乘数,nb存储乘数,nc存储积
    fill(na,na+L,0);fill(nb,nb+L,0);fill(nc,nc+L,0);//将na,nb,nc都置为0
    for(int i=La-1;i>=0;i--) na[La-i]=a[i]-'0';//将字符串表示的大整形数转成i整形数组表示的大整形数
    for(int i=Lb-1;i>=0;i--) nb[Lb-i]=b[i]-'0';
    for(int i=1;i<=La;i++)
        for(int j=1;j<=Lb;j++)
        nc[i+j-1]+=na[i]*nb[j];//a的第i位乘以b的第j位为积的第i+j-1位(先不考虑进位)
    for(int i=1;i<=La+Lb;i++)
        nc[i+1]+=nc[i]/10,nc[i]%=10;//统一处理进位
    if(nc[La+Lb]) s+=nc[La+Lb]+'0';//判断第i+j位上的数字是不是0
    for(int i=La+Lb-1;i>=1;i--)
        s+=nc[i]+'0';//将整形数组转成字符串
    return s;
}
\end{lstlisting}

\subsection{高精度a乘低精度b}


\begin{lstlisting}[language={[ANSI]C},numbers=left]

string mul(string a,int b)//高精度a乘单精度b
{
    const int L=100005;
    int na[L];
    string ans;
    int La=a.size();
    fill(na,na+L,0);
    for(int i=La-1;i>=0;i--) na[La-i-1]=a[i]-'0';
    int w=0;
    for(int i=0;i<La;i++) na[i]=na[i]*b+w,w=na[i]/10,na[i]=na[i]%10;
    while(w) na[La++]=w%10,w/=10;
    La--;
    while(La>=0) ans+=na[La--]+'0';
    return ans;
}
\end{lstlisting}

\subsection{高精除高精+高精除低精}


\begin{lstlisting}[language={[ANSI]C},numbers=left]

//除法
int sub(int *a,int *b,int La,int Lb)
{
    if(La<Lb) return -1;//如果a小于b,则返回-1
    if(La==Lb)
    {
        for(int i=La-1;i>=0;i--)
            if(a[i]>b[i]) break;
            else if(a[i]<b[i]) return -1;//如果a小于b,则返回-1

    }
    for(int i=0;i<La;i++)//高精度减法
    {
        a[i]-=b[i];
        if(a[i]<0) a[i]+=10,a[i+1]--;
    }
    for(int i=La-1;i>=0;i--)
        if(a[i]) return i+1;//返回差的位数
    return 0;//返回差的位数

}
string div(string n1,string n2,int nn)
//n1,n2是字符串表示的被除数,除数,nn是选择返回商还是余数
{
    const int L=1e5;
    string s,v;//s存商,v存余数
     int a[L],b[L],r[L],La=n1.size(),Lb=n2.size(),i,tp=La;
     //a,b是整形数组表示被除数,除数,tp保存被除数的长度
     fill(a,a+L,0);fill(b,b+L,0);fill(r,r+L,0);//数组元素都置为0
     for(i=La-1;i>=0;i--) a[La-1-i]=n1[i]-'0';
     for(i=Lb-1;i>=0;i--) b[Lb-1-i]=n2[i]-'0';
     if(La<Lb || (La==Lb && n1<n2)) {
            //cout<<0<<endl;
     return n1;}//如果a<b,则商为0,余数为被除数
     int t=La-Lb;//除被数和除数的位数之差
     for(int i=La-1;i>=0;i--)//将除数扩大10^t倍
        if(i>=t) b[i]=b[i-t];
        else b[i]=0;
     Lb=La;
     for(int j=0;j<=t;j++)
     {
         int temp;
         while((temp=sub(a,b+j,La,Lb-j))>=0)//如果被除数比除数大继续减
         {
             La=temp;
             r[t-j]++;
         }
     }
     for(i=0;i<L-10;i++) r[i+1]+=r[i]/10,r[i]%=10;//统一处理进位
     while(!r[i]) i--;//将整形数组表示的商转化成字符串表示的
     while(i>=0) s+=r[i--]+'0';
     //cout<<s<<endl;
     i=tp;
     while(!a[i]) i--;//将整形数组表示的余数转化成字符串表示的</span>
     while(i>=0) v+=a[i--]+'0';
     if(v.empty()) v="0";
     //cout<<v<<endl;
     if(nn==1) return s;//返回商
     if(nn==2) return v;//返回余数
}
string div(string a,int b)//高精度a除以单精度b
{
    string r,ans;
    int d=0;
    if(a=="0") return a;//特判
    for(int i=0;i<a.size();i++)
    {
            r+=(d*10+a[i]-'0')/b+'0';//求出商
            d=(d*10+(a[i]-'0'))%b;//求出余数
    }
    int p=0;
    for(int i=0;i<r.size();i++)
    if(r[i]!='0') {p=i;break;}
    return r.substr(p);
}
\end{lstlisting}

\subsection{高精度幂}

使用了FFT,时复(nlog(n)log(n))
\begin{lstlisting}[language={[ANSI]C},numbers=left]

//高精度幂(nlog(n)log(n))
#define L(x) (1 << (x))
const double PI = acos(-1.0);
const int Maxn = 133015;
double ax[Maxn], ay[Maxn], bx[Maxn], by[Maxn];
char sa[Maxn/2],sb[Maxn/2];
int sum[Maxn];
int x1[Maxn],x2[Maxn];
int revv(int x, int bits)
{
    int ret = 0;
    for (int i = 0; i < bits; i++)
    {
        ret <<= 1;
        ret |= x & 1;
        x >>= 1;
    }
    return ret;
}
void fft(double * a, double * b, int n, bool rev)
{
    int bits = 0;
    while (1 << bits < n) ++bits;
    for (int i = 0; i < n; i++)
    {
        int j = revv(i, bits);
        if (i < j)
            swap(a[i], a[j]), swap(b[i], b[j]);
    }
    for (int len = 2; len <= n; len <<= 1)
    {
        int half = len >> 1;
        double wmx = cos(2 * PI / len), wmy = sin(2 * PI / len);
        if (rev) wmy = -wmy;
        for (int i = 0; i < n; i += len)
        {
            double wx = 1, wy = 0;
            for (int j = 0; j < half; j++)
            {
                double cx = a[i + j], cy = b[i + j];
                double dx = a[i + j + half], dy = b[i + j + half];
                double ex = dx * wx - dy * wy, ey = dx * wy + dy * wx;
                a[i + j] = cx + ex, b[i + j] = cy + ey;
                a[i + j + half] = cx - ex, b[i + j + half] = cy - ey;
                double wnx = wx * wmx - wy * wmy, wny = wx * wmy + wy * wmx;
                wx = wnx, wy = wny;
            }
        }
    }
    if (rev)
    {
        for (int i = 0; i < n; i++)
            a[i] /= n, b[i] /= n;
    }
}
int solve(int a[],int na,int b[],int nb,int ans[])
{
    int len = max(na, nb), ln;
    for(ln=0; L(ln)<len; ++ln);
    len=L(++ln);
    for (int i = 0; i < len ; ++i)
    {
        if (i >= na) ax[i] = 0, ay[i] =0;
        else ax[i] = a[i], ay[i] = 0;
    }
    fft(ax, ay, len, 0);
    for (int i = 0; i < len; ++i)
    {
        if (i >= nb) bx[i] = 0, by[i] = 0;
        else bx[i] = b[i], by[i] = 0;
    }
    fft(bx, by, len, 0);
    for (int i = 0; i < len; ++i)
    {
        double cx = ax[i] * bx[i] - ay[i] * by[i];
        double cy = ax[i] * by[i] + ay[i] * bx[i];
        ax[i] = cx, ay[i] = cy;
    }
    fft(ax, ay, len, 1);
    for (int i = 0; i < len; ++i)
        ans[i] = (int)(ax[i] + 0.5);
    return len;
}
string mul(string sa,string sb)
{
    int l1,l2,l;
    int i;
    string ans;
    memset(sum, 0, sizeof(sum));
    l1 = sa.size();
    l2 = sb.size();
    for(i = 0; i < l1; i++)
        x1[i] = sa[l1 - i - 1]-'0';
    for(i = 0; i < l2; i++)
        x2[i] = sb[l2-i-1]-'0';
    l = solve(x1, l1, x2, l2, sum);
    for(i = 0; i<l || sum[i] >= 10; i++) // 进位
    {
        sum[i + 1] += sum[i] / 10;
        sum[i] %= 10;
    }
    l = i;
    while(sum[l] <= 0 && l>0)    l--; // 检索最高位
    for(i = l; i >= 0; i--)    ans+=sum[i] + '0'; // 倒序输出
    return ans;
}
string Pow(string a,int n)
{
    if(n==0) return 1;
    if(n==1) return a;
    if(n&1) return mul(Pow(a,n-1),a);
    string ans=Pow(a,n/2);
    return mul(ans,ans);
}
\end{lstlisting}

\subsection{阶层}


\begin{lstlisting}[language={[ANSI]C},numbers=left]

//阶层
string fac(int n)
{
    const int L=100005;
    int a[L];
    string ans;
    if(n==0) return "1";
    fill(a,a+L,0);
    int s=0,m=n;
    while(m) a[++s]=m%10,m/=10;
    for(int i=n-1;i>=2;i--)
    {
        int w=0;
        for(int j=1;j<=s;j++) a[j]=a[j]*i+w,w=a[j]/10,a[j]=a[j]%10;
        while(w) a[++s]=w%10,w/=10;
    }
    while(!a[s]) s--;
    while(s>=1) ans+=a[s--]+'0';
    return ans;
}
\end{lstlisting}

\subsection{求最大公约数}


\begin{lstlisting}[language={[ANSI]C},numbers=left]

//gcd
string add(string a,string b)
{
    const int L=1e5;
    string ans;
    int na[L]={0},nb[L]={0};
    int la=a.size(),lb=b.size();
    for(int i=0;i<la;i++) na[la-1-i]=a[i]-'0';
    for(int i=0;i<lb;i++) nb[lb-1-i]=b[i]-'0';
    int lmax=la>lb?la:lb;
    for(int i=0;i<lmax;i++) na[i]+=nb[i],na[i+1]+=na[i]/10,na[i]%=10;
    if(na[lmax]) lmax++;
    for(int i=lmax-1;i>=0;i--) ans+=na[i]+'0';
    return ans;
}
string mul(string a,string b)
{
    const int L=1e5;
    string s;
    int na[L],nb[L],nc[L],La=a.size(),Lb=b.size();//na存储被乘数,nb存储乘数,nc存储积
    fill(na,na+L,0);fill(nb,nb+L,0);fill(nc,nc+L,0);//将na,nb,nc都置为0
    for(int i=La-1;i>=0;i--) na[La-i]=a[i]-'0';//将字符串表示的大整形数转成i整形数组表示的大整形数
    for(int i=Lb-1;i>=0;i--) nb[Lb-i]=b[i]-'0';
    for(int i=1;i<=La;i++)
        for(int j=1;j<=Lb;j++)
        nc[i+j-1]+=na[i]*nb[j];//a的第i位乘以b的第j位为积的第i+j-1位(先不考虑进位)
    for(int i=1;i<=La+Lb;i++)
        nc[i+1]+=nc[i]/10,nc[i]%=10;//统一处理进位
    if(nc[La+Lb]) s+=nc[La+Lb]+'0';//判断第i+j位上的数字是不是0
    for(int i=La+Lb-1;i>=1;i--)
        s+=nc[i]+'0';//将整形数组转成字符串
    return s;
}
int sub(int *a,int *b,int La,int Lb)
{
    if(La<Lb) return -1;//如果a小于b,则返回-1
    if(La==Lb)
    {
        for(int i=La-1;i>=0;i--)
            if(a[i]>b[i]) break;
            else if(a[i]<b[i]) return -1;//如果a小于b,则返回-1

    }
    for(int i=0;i<La;i++)//高精度减法
    {
        a[i]-=b[i];
        if(a[i]<0) a[i]+=10,a[i+1]--;
    }
    for(int i=La-1;i>=0;i--)
        if(a[i]) return i+1;//返回差的位数
    return 0;//返回差的位数

}
string div(string n1,string n2,int nn)//n1,n2是字符串表示的被除数,除数,nn是选择返回商还是余数
{
    const int L=1e5;
    string s,v;//s存商,v存余数
     int a[L],b[L],r[L],La=n1.size(),Lb=n2.size(),i,tp=La;//a,b是整形数组表示被除数,除数,tp保存被除数的长度
     fill(a,a+L,0);fill(b,b+L,0);fill(r,r+L,0);//数组元素都置为0
     for(i=La-1;i>=0;i--) a[La-1-i]=n1[i]-'0';
     for(i=Lb-1;i>=0;i--) b[Lb-1-i]=n2[i]-'0';
     if(La<Lb || (La==Lb && n1<n2)) {
            //cout<<0<<endl;
     return n1;}//如果a<b,则商为0,余数为被除数
     int t=La-Lb;//除被数和除数的位数之差
     for(int i=La-1;i>=0;i--)//将除数扩大10^t倍
        if(i>=t) b[i]=b[i-t];
        else b[i]=0;
     Lb=La;
     for(int j=0;j<=t;j++)
     {
         int temp;
         while((temp=sub(a,b+j,La,Lb-j))>=0)//如果被除数比除数大继续减
         {
             La=temp;
             r[t-j]++;
         }
     }
     for(i=0;i<L-10;i++) r[i+1]+=r[i]/10,r[i]%=10;//统一处理进位
     while(!r[i]) i--;//将整形数组表示的商转化成字符串表示的
     while(i>=0) s+=r[i--]+'0';
     //cout<<s<<endl;
     i=tp;
     while(!a[i]) i--;//将整形数组表示的余数转化成字符串表示的</span>
     while(i>=0) v+=a[i--]+'0';
     if(v.empty()) v="0";
     //cout<<v<<endl;
     if(nn==1) return s;
     if(nn==2) return v;
}
bool judge(string s)//判断s是否为全0串
{
    for(int i=0;i<s.size();i++)
        if(s[i]!='0') return false;
    return true;
}
string gcd(string a,string b)//求最大公约数
{
    string t;
    while(!judge(b))//如果余数不为0,继续除
    {
        t=a;//保存被除数的值
        a=b;//用除数替换被除数
        b=div(t,b,2);//用余数替换除数
    }
    return a;
}
\end{lstlisting}

\subsection{取模(高精)}


\begin{lstlisting}[language={[ANSI]C},numbers=left]

//取模
int mod(string a,int b)//高精度a除以单精度b
{
    int d=0;
    for(int i=0;i<a.size();i++) d=(d*10+(a[i]-'0'))%b;//求出余数
    return d;
}  
\end{lstlisting}

\subsection{进制转换(将字符串表示的10进制大整数转换为m进制的大整数)}


\begin{lstlisting}[language={[ANSI]C},numbers=left]

//进制转换
//将字符串表示的10进制大整数转换为m进制的大整数
//并返回m进制大整数的字符串
bool judge(string s)//判断串是否为全零串
{
    for(int i=0;i<s.size();i++)
        if(s[i]!='0') return 1;
    return 0;
}
string solve(string s,int n,int m)//n进制转m进制只限0-9进制,若涉及带字母的进制,稍作修改即可
{
    string r,ans;
    int d=0;
    if(!judge(s)) return "0";//特判
    while(judge(s))//被除数不为0则继续
    {
        for(int i=0;i<s.size();i++)
        {
            r+=(d*n+s[i]-'0')/m+'0';//求出商
            d=(d*n+(s[i]-'0'))%m;//求出余数
        }
       s=r;//把商赋给下一次的被除数
       r="";//把商清空
        ans+=d+'0';//加上进制转换后数字
        d=0;//清空余数
    }
    reverse(ans.begin(),ans.end());//倒置下
    return ans;
}
\end{lstlisting}


\section{计算几何}

\subsection{基础模板}
\begin{lstlisting}[language={[ANSI]C},numbers=left]
//点和线的表示
typedef long double LD;
struct PII{
	int x,y;
	bool operator<(const Point &o)const{
		if(x==o.x) return y<o.y;
		return x<o.x;
	}
};
struct Point{
	double x,y;
	bool operator<(const Point &o)const{
		if(x==o.x) return y<o.y;
		return x<o.x;
	}
};
struct Line{
	Point st,ed;
};
//判正负
int sign(double a){
	if(fabs(a)<=eps) return 0;
	return a>0?1:-1;
}
//比较大小
int dcmp(double a,double b){
	if(fabs(a-b)<eps) return 0;
	return a>b?1:-1;
}
\end{lstlisting}

\subsection{点积}
\begin{lstlisting}[language={[ANSI]C},numbers=left]
double Dot(Point a, Point b) {
	return a.x*b.x+a.y*b.y;
}
\end{lstlisting}

\subsection{叉积}

\begin{lstlisting}[language={[ANSI]C},numbers=left]
double cross(Point a,Point b){ 
	return a.x*b.y-a.y*b.x;
}
\end{lstlisting}

\subsection{模长}

\begin{lstlisting}[language={[ANSI]C},numbers=left]
double ABS(Point a){
	return sqrt(a.x*a.x+a.y*a.y);
}
double norm(Point a){
    return a.x*a.x+a.y*a.y;
}
\end{lstlisting}

\subsection{两个向量是否同象限}

\begin{lstlisting}[language={[ANSI]C},numbers=left]
bool same_quadrant(Point v,Point p) {
	LD a=v.x, b=v.y, c=p.x, d=p.y;
	int aa=sign(a), bb=sign(b);
	int cc=sign(c), dd=sign(d);
	return aa*cc>=0 && bb*dd>=0;
}
\end{lstlisting}

\subsection{两向量是否共线}

\begin{lstlisting}[language={[ANSI]C},numbers=left]
int dcmp(double x) {
	if (fabs(x)<eps) return 0;
	else if (x<0) return -1;
	else return 1;
}
double cross(Point a,Point b){ 
	return a.x*b.y-a.y*b.x;
}
bool on_line(Point a, Point b) {	//a和b是否共线 
	return dcmp(cross(a,b))==0;
}
\end{lstlisting}

\subsection{两向量是否垂直}

\begin{lstlisting}[language={[ANSI]C},numbers=left]
double Dot(Point a, Point b) {
	return a.x*b.x+a.y*b.y;
}
bool is_vertical(Point a, Point b) {
	return dcmp(Dot(a,b))==0;
}
\end{lstlisting}

\subsection{两个向量是否同方向}

\begin{lstlisting}[language={[ANSI]C},numbers=left]
int same_direction(Point v, Point p) {    //判断向量v和向量p是否共线且同向
	if (on_line(v, p) && same_quadrant(v,p)) return 1;    //同向
	else if(on_line(v, p) && !same_quadrant(v,p)) return -1;    //反向
	return 0;    //不共线
}
\end{lstlisting}

\subsection{点旋转}

\begin{lstlisting}[language={[ANSI]C},numbers=left]
//点逆时针旋转a度后的坐标
Point rotate1(Point p,double a){
	return {p.x*cos(a)-p.y*sin(a),p.x*sin(a)+p.y*cos(a)};
}
//点顺时针旋转a度后的坐标
Point rotate2(Point p,double a){
	return {p.x*cos(a)+p.y*sin(a),-p.x*sin(a)+p.y*cos(a)};
}
\end{lstlisting}

\subsection{两点之间的距离}

\begin{lstlisting}[language={[ANSI]C},numbers=left]
double getdis(Point a,Point b) {
	return hypot(a.x-b.x,a.y-b.y);
}
\end{lstlisting}

\subsection{直线}
\subsubsection{直线和线段是否相交}
\begin{lstlisting}[language={[ANSI]C},numbers=left]
//直线a是否经过线段b 
bool on_segment(Line a,Line b){
	point q1=a.st,q2=a.ed;
	if(sign(area(q1,q2,b.st))*sign(area(q1,q2,b.ed))>0) return 0;
	return 1;
}
\end{lstlisting}
\subsubsection{直线和直线的交点}

\begin{lstlisting}[language={[ANSI]C},numbers=left]
//求两直线的交点
Point get_line_intersection(Point p, Point v, Point q, Point w){
    Point u = p - q;
    LD t = cross(w, u) / cross(v, w);
    return {p.x + t * v.x, p.y + t * v.y};
}
Point get_line_intersection(Line& a, Line& b){
    return get_line_intersection(a.st, a.ed - a.st, b.st, b.ed - b.st);
}
\end{lstlisting}

\subsubsection{判断b和c的交点是否在直线右面}

\begin{lstlisting}[language={[ANSI]C},numbers=left]
bool on_right(Line& a, Line& b, Line& c){
    auto o = get_line_intersection(b, c);
    return sign(area(a.st, a.ed, o)) < 0;
}
\end{lstlisting}

\subsubsection{得到直线的角度}

\begin{lstlisting}[language={[ANSI]C},numbers=left]
double get_angle(const Line &a){
	return atan2(a.ed.y-a.st.y,a.ed.x-a.st.x);
}
\end{lstlisting}

\subsubsection{求点关于直线的对称点}

\begin{lstlisting}[language={[ANSI]C},numbers=left]
Point point_line(Line l, Point p) { //点p关于直线l的对称点
	Point p1 = l.st;
	Point p2 = l.ed;
	double _x, _y;
	if(p1.x - p2.x == 0) { //l斜率不存在
		_x = 2 * p1.x - p.x;
		_y = p.y;
		return Point{_x,_y};
	} else if(p1.y - p2.y == 0) { //l斜率为0
		_x = p.x;
		_y = 2 * p1.y - p.y;
		return Point{_x,_y};
	} else {
		double k1 = (p1.y - p2.y) / (p1.x - p2.x);
		double b1 = p1.y - k1 * p1.x;
		double k2 = -1 / k1;
		double b2 = p.y - k2 * p.x;
		_x = (b2 - b1) / (k1 - k2);
		_y = k2 * _x + b2;
		return Point{2 * _x - p.x, 2 * _y - p.y};
	}
}
\end{lstlisting}

\subsubsection{点到直线的距离}

\begin{lstlisting}[language={[ANSI]C},numbers=left]
double PLDis(Point a,Line s) {	//点到直线的距离
    double A=s.st.y-s.ed.y;
    double B=s.ed.x-s.st.x;
    double C=(s.st.x-s.ed.x)*s.st.y-(s.st.y-s.ed.y)*s.st.x;
    return fabs(A*a.x+B*a.y+C)/sqrt((A*A+B*B));
}
\end{lstlisting}

\subsubsection{点在直线的投影}

\begin{lstlisting}[language={[ANSI]C},numbers=left]
//点p在直线s上的投影
Point Projection(Point p,Line s) {
    Point alp=p-s.st;
    Point beta=s.ed-s.st;
    double res=Dot(alp,beta)/norm(beta);	//norm(): 模长的平方
    return s.st+(res*beta);
}
\end{lstlisting}

\subsection{求多边形的面积}

\begin{lstlisting}[language={[ANSI]C},numbers=left]
vector<Point> polygon;
double get_Area(vector<Point> polygon) {
	double ans=0;
	int n=(int)polygon.size();
	for(int i=0; i<n; i++) {
		ans+=cross(polygon[i],polygon[(i+1)%n]);
	}
	return fabs(ans/2);
}
\end{lstlisting}

\subsection{线段}
\subsubsection{判断一个点是否在一个线段上}

\begin{lstlisting}[language={[ANSI]C},numbers=left]
bool on_segment(Point p, Point a, Point b){
	return sign(cross(p - a, p - b)) == 0 && sign(dot(p - a, p - b)) <= 0;
}
\end{lstlisting}

\subsection{判断两个线段是否相交}
\begin{lstlisting}[language={[ANSI]C},numbers=left]
bool segment_intersection(Point a1, Point a2, Point b1, Point b2){
	double c1 = cross(a2 - a1, b1 - a1), c2 = cross(a2 - a1, b2 - a1);
	double c3 = cross(b2 - b1, a2 - b1), c4 = cross(b2 - b1, a1 - b1);
	return sign(c1) * sign(c2) <= 0 && sign(c3) * sign(c4) <= 0;
}
\end{lstlisting}

\subsubsection{判断线和线段是否相交}

\begin{lstlisting}[language={[ANSI]C},numbers=left]
bool line_segment_intersection(Line a,Line b){
	if(on_line(a.p2-a.p1,b.p2-b.p1)){
		if(on_line(a.p2-a.p1,b.p1-a.p1)) return 1;
		else return 0;
	} 
	Point o=get_line_intersection(a,b);
	if(on_segment(o,b)) return 1;
	else return 0;
}
\end{lstlisting}

\subsection{圆}

\subsubsection{直线和圆的交点}

\begin{lstlisting}[language={[ANSI]C},numbers=left]
int CCL(Line s,Point o,double r,Point &o1,Point &o2) {
    Point x=Projection(o,s);
    double dis=PLDis(o,s);
    if(dis>r) { //距离>r没有交点 
        return 0;
    }
    if(dis==r) { //只有一个交点
        o1=x;
        return 1;
    }
    double beta=sqrt(r*r-dis*dis);//勾股定理
    Point pp=s.ed-s.st;
    pp=pp/pp.ABS();//单位向量
    Point ans1=x-beta*pp;
    Point ans2=x+beta*pp;
    o1=ans1;
    o2=ans2;
    return 2;
}
\end{lstlisting}



\subsection{凸包}

求包含所有点的最小周长多边形

\begin{lstlisting}[language={[ANSI]C},numbers=left]

//极角排序比较函数
bool cmp(PDD a,PDD b){
    a=a-bas; b=b-bas;
	double ag1=atan2(a.y,a.x),ag2=atan2(b.y,b.x);
    if(ag1==ag2) a.x<b.x;
	else return ag1<ag2;
}
//二维凸包
void get_convex(){
	sort(q+1,q+1+n,cmp);
	stk[++top]=q[1];
	stk[++top]=q[2];
	for(int i=3;i<=n;i++){
		while(top>=2 && area(stk[top-1],stk[top],q[i])<=0) --top;
		stk[++top]=q[i];
	}
	return ;
}
\end{lstlisting}

\subsection{三维凸包}

求将三维空间中的n个点包含进去所需要的最小多边形面积
\begin{lstlisting}[language={[ANSI]C},numbers=left]

#include <bits/stdc++.h>
#define ios ios::sync_with_stdio(0);cin.tie(0);cout.tie(0)
#define endl '\n'
using namespace std;
typedef long double LD;
const LD eps=1e-10;
const int N=110;
const double PI=acos(-1);
double rand_eps(){
	return ((double)rand()/RAND_MAX-0.5)*eps;
}
struct Point{
	double x,y,z;
	void shake(){	//微小扰动
		x+=rand_eps();
		y+=rand_eps();
		z+=rand_eps();
	}
	Point operator - (Point a){
		return {x-a.x,y-a.y,z-a.z};
	}
	Point operator + (Point a){
		return {x+a.x,y+a.y,z+a.z};
	}
	Point operator * (Point a){	//向量叉乘
		return {y * a.z - z * a.y, z * a.x - x * a.z, x * a.y - y * a.x};
	}
	double operator & (Point a){	//向量点积
		return x*a.x+y*a.y+z*a.z;
	}
	double len(){	//向量模长
		return sqrt(x*x+y*y+z*z);
	}
}q[N];
struct Plane{
	int v[3];
	Point norm(){
		return (q[v[1]]-q[v[0]])*(q[v[2]]-q[v[0]]);	//返回法向量
	}
	double area(){
		return norm().len()/2;	//面积
	}
	bool above(Point a){
		return ((a-q[v[0]]) & norm()) >= 0;	//返回一个点是否在一个平面上方,也就是平面能不能被照到
	}
}pl[N],bk[N];
bool g[N][N];
int n,tot;
void get_convex(){
	pl[++tot]={1,2,3};	//放进去前3个点组成的两个平面(正反)
	pl[++tot]={1,3,2};
	for(int i=4;i<=n;i++){
		int cnt=0;	//把更新后的平面放进去备份数组
		for(int j=1;j<=tot;j++){
			bool flag=pl[j].above(q[i]);	//表示q[i]照到了第j个平面
			if(!flag) bk[++cnt]=pl[j];	//没有照到
			for(int k=0;k<3;k++){
				g[pl[j].v[k]][pl[j].v[(k+1)%3]]=flag;	//标记每一条边是否照到
			}
		}
		for(int j=1;j<=tot;j++){
			for(int k=0;k<3;k++){
				int a=pl[j].v[k],b=pl[j].v[(k+1)%3];	//两点的编号
				if(g[a][b] && !g[b][a]) bk[++cnt]={a,b,i};	//正着可以照到,反着照不到,把照到的边扔掉
			}
		}
		tot=cnt;	//更新面的数量
		for(int i=1;i<=tot;i++) pl[i]=bk[i];	//更新所有的面
	}
}
int main()
{
	ios;
	cin>>n;
	for(int i=1;i<=n;i++){
		cin>>q[i].x>>q[i].y>>q[i].z;
		q[i].shake();
	}
	get_convex();
	double ans=0;
	for(int i=1;i<=tot;i++){
		ans+=pl[i].area();
	}
	cout<<fixed<<setprecision(6)<<ans<<endl;
	
	return 0;
 } 
\end{lstlisting}

\subsection{半平面交}

求包含所有点的最小周长多边形

\begin{lstlisting}[language={[ANSI]C},numbers=left]

//极角排序比较函数
bool cmp(PDD a,PDD b){
    a=a-bas; b=b-bas;
	double ag1=atan2(a.y,a.x),ag2=atan2(b.y,b.x);
    if(ag1==ag2) a.x<b.x;
	else return ag1<ag2;
}
//二维凸包
void get_convex(){
	sort(q+1,q+1+n,cmp);
	stk[++top]=q[1];
	stk[++top]=q[2];
	for(int i=3;i<=n;i++){
		while(top>=2 && area(stk[top-1],stk[top],q[i])<=0) --top;
		stk[++top]=q[i];
	}
	return ;
}
\end{lstlisting}

\subsection{最小圆覆盖}


\begin{lstlisting}[language={[ANSI]C},numbers=left]

#include <bits/stdc++.h>
#define ios ios::sync_with_stdio(0);cin.tie(0);cout.tie(0)
#define endl '\n'
#define x first
#define y second
using namespace std;
typedef long double LD;
const LD eps=1e-12;
const int N=101000;
const double PI=acos(-1);
//点和线的表示
typedef long double LD;
typedef pair<double,double> PDD;
struct Circle {
	PDD p;

	double r;
};
//重载运算符"-"
PDD operator-(const PDD &a,const PDD &b) {
	return {a.x-b.x,a.y-b.y};
}
PDD operator+(const PDD &a,const PDD &b) {
	return {a.x+b.x,a.y+b.y};
}
PDD operator/ (const PDD &a,double t) {
	return {a.x/t,a.y/t};
}
//求向量a和b的叉积
double cross(PDD a,PDD b) {
	return a.x*b.y-a.y*b.x;
}
//求向量ab和向量ac的叉积,也就是abc三角形的面积,顺时针为负,逆时针为正
double area(PDD a,PDD b,PDD c) {
	return cross(b-a,c-a);
}
//判正负
int sign(double a) {
	if(fabs(a)<=eps) return 0;
	return a>0?1:-1;
}
//比较大小
int fcmp(double a,double b) {
	if(fabs(a-b)<eps) return 0;
	return a>b?1:-1;
}
PDD get_line_intersection(PDD p, PDD v, PDD q, PDD w) {
	PDD u = p - q;
	LD t = cross(w, u) / cross(v, w);
	return {p.x + t * v.x, p.y + t * v.y};
}
//点逆时针旋转a度后的坐标
PDD rotate(PDD p,double a) {
	return {p.x*cos(a)-p.y*sin(a),p.x*sin(a)+p.y*cos(a)};
}
pair<PDD,PDD> getline(PDD a,PDD b) {
	return {(a+b)/2,rotate(b-a,PI/2)};
}
double getdis(PDD a,PDD b) {
	return sqrt(pow(b.y-a.y,2)+pow(b.x-a.x,2));
}
Circle get_circle(PDD a,PDD b,PDD c) {
	auto u=getline(a,b),v=getline(a,c);
	auto p=get_line_intersection(u.x,u.y,v.x,v.y);
	return {p,getdis(p,a)};
}
PDD q[N];
int n;
Circle get_Circle() {
	random_shuffle(q+1,q+1+n);
	Circle c;
	c.p=q[1];
	c.r=0;
	for(int i=2; i<=n; i++) {
		if(fcmp(c.r,getdis(c.p,q[i]))<0) {
			c= {q[i],0};
			for(int j=1; j<i; j++) {
				if(fcmp(c.r,getdis(c.p,q[j]))<0) {
					c= {(q[i]+q[j])/2,getdis(q[i],q[j])/2};
					for(int k=1; k<j; k++) {
						if(fcmp(c.r,getdis(c.p,q[k]))<0) {
							c=get_circle(q[i],q[j],q[k]);
						}
					}
				}
			}
		}
	}
	return c;
}

int main() {
	ios;
	cin>>n;
	for(int i=1; i<=n; i++) cin>>q[i].x>>q[i].y;
	Circle c;
	c=get_Circle();
	cout<<fixed<<setprecision(10)<<c.r<<endl;
	cout<<fixed<<setprecision(10)<<c.p.x<<" "<<c.p.y<<endl;
	return 0;
}
\end{lstlisting}

\section{数据结构}

\subsection{单调栈}

\subsubsection{求某组数以其中某一个数字为最小值的最大延伸区间}

\begin{lstlisting}[language={[ANSI]C},numbers=left]

int rmi[N],rmx[N];
void largest(int *arr,int n) {
    stack<int> stmin,stmax;
    for(int i=1;i<=n;i++){
    	while(!stmin.empty() && arr[i]<arr[stmin.top()]){
    		rmi[stmin.top()]=i-1;
    		stmin.pop();
		}
    	stmin.push(i);
    	while(!stmax.empty() && arr[i]>arr[stmax.top()]) {
    		rmx[stmax.top()]=i-1;
    		stmax.pop();
		}
		stmax.push(i);
	}
	while(!stmin.empty()) {
		rmi[stmin.top()]=n;
		stmin.pop();
	}
	while(!stmax.empty()) {
		rmx[stmax.top()]=n;
		stmax.pop();
	}
	return ;
}
\end{lstlisting}

\subsubsection{求一个柱状图的最大矩形面积}

维护栈内元素单调递增,如果来一个元素比栈顶元素小,那么栈顶元素右边就没法再增大,所以栈顶位置的最大面积就可以直接算了,s=(i-st.top())\*(heights[st.top()]),需要注意的是初始数组最后添加一个能把前面都小的值0
\begin{lstlisting}[language={[ANSI]C},numbers=left]

int largestRectangleArea(vector<int>& heights) {
    heights.push_back(0);
    stack<int> st;
    int sz=heights.size();
    int ans=0;
    for(int i=0; i<sz; i++) {
        while(!st.empty() && heights[st.top()]>heights[i]) {
            int cur=heights[st.top()];
            st.pop();
            if(st.empty()) ans=max(ans,i*cur);	//这里记得特判
            else ans=max(ans,(i-st.top()-1)*cur);
        }
        st.push(i);
    }
    return ans;
}
\end{lstlisting}

\subsection{单调队列}

\subsubsection{区间(长度固定)最值问题}


\begin{lstlisting}[language={[ANSI]C},numbers=left]

int mi[N],mx[N];
void GET(int *arr,int n,int k){
	deque<int> dqmin,dqmax;
	for(int i=1;i<=n;i++){
		while(!dqmax.empty() && arr[i]>=arr[dqmax.back()]) dqmax.pop_back();
		dqmax.push_back(i);
		while(!dqmin.empty() && arr[i]<=arr[dqmin.back()]) dqmin.pop_back();
		dqmin.push_back(i);
		if(i>=k){
			mx[i-k+1]=arr[dqmax.front()];
			mi[i-k+1]=arr[dqmin.front()];
		}
		if(i-dqmax.front()+1>=k) dqmax.pop_front();
		if(i-dqmin.front()+1>=k) dqmin.pop_front();
	}
	return ;
}
\end{lstlisting}

\subsubsection{最大值减最小值小于等于k的子区间数量}

求一个数组中区间最大值减去最小值<=k的所有子区间个数
\begin{lstlisting}[language={[ANSI]C},numbers=left]

ll  GETNUM(vector<ll> nums,int k){
	deque<ll> dqmin,dqmax;
	ll sz=nums.size(),j=0,ans=0;
	for(int i=0;i<sz;i++){
		while(j<sz){
            //维护从大到小的deque
			while(!dqmax.empty() && nums[j]>=nums[dqmax.back()]) dqmax.pop_back();
			dqmax.push_back(j);
            //维护从小到大的deque            
			while(!dqmin.empty() && nums[j]<=nums[dqmin.back()]) dqmin.pop_back();
			dqmin.push_back(j);
            //不满足条件就break出去
			if(nums[dqmax.front()]-nums[dqmin.front()]>k) break;
			j++;
		}
        //L指针右走时要看看需不需要更新dq,如果dq队头是这个元素就需要pop出去
		if(dqmax.front()==i) dqmax.pop_front();
		if(dqmin.front()==i) dqmin.pop_front();
        //这个时候R-L就是以L开头的所有最大减最小<=k的区间个数
		ans+=j-i;
	}
	return ans;
}
\end{lstlisting}

\subsection{并查集}

\subsubsection{按秩合并}

\begin{lstlisting}[language={[ANSI]C},numbers=left]

void remerge(int a,int b,int c) {
	int x=find(a), y=find(b);
	if(x==y) return ;
	if(dep[x]<dep[y]) fa[x]=y;
	else {
		fa[y]=x;
		if(dep[x]==dep[y]) dep[x]++;
	}
}
\end{lstlisting}

\subsubsection{关押罪犯}

\begin{lstlisting}[language={[ANSI]C},numbers=left]

#include <bits/stdc++.h>
#define ios ios::sync_with_stdio(0);cin.tie(0);cout.tie(0)
using namespace std;
const int N=1e6+100;
struct node{
	int u,v,w;
}arr[100100];
int n,m;
int fa[N*2];
int find(int x){
	if(x==fa[x]) return x;
	else return fa[x]=find(fa[x]);
}
void remerge(int x,int y){
	int fx=find(x),fy=find(y);
	if(fx!=fy) fa[fx]=fy;
}
bool cmp(node a,node b){
	return a.w>b.w;
}
int main(){
	ios;
	cin>>n>>m;
	for(int i=1;i<=m;i++) cin>>arr[i].u>>arr[i].v>>arr[i].w;
	sort(arr+1,arr+1+m,cmp);
	for(int i=1;i<=2*n;i++) fa[i]=i;
	for(int i=1;i<=m;i++){
		int x=arr[i].u,y=arr[i].v;
		if(find(x)==find(y) || find(x+n)==find(y+n)){
			cout<<arr[i].w<<'\n';
			return 0;
		}
		remerge(x+n,y);
        remerge(x,y+n);
	}
	cout<<0<<'\n';
	return 0;
}


\end{lstlisting}

\subsubsection{食物链}

dis[i]是i指向其父亲的向量模长,向量的方向也是有意义的,模长决定了i属于哪个集合,而方向表示i是被父亲吃还是吃父亲
\begin{lstlisting}[language={[ANSI]C},numbers=left]

#include <bits/stdc++.h>
#define ios ios::sync_with_stdio(0);cin.tie(0);cout.tie(0)
#define endl '\n'
using namespace std;
const int N=5e4+100;
int fa[N],dis[N];
int n,k;
int find(int x){
	if(x==fa[x]) return x;
	else{
		int tmp=fa[x];
		fa[x]=find(fa[x]);
		dis[x]=(dis[x]+dis[tmp])%3;
		return fa[x];
	}
}

int main()
{
	ios;
	cin>>n>>k;
	for(int i=1;i<=n;i++) fa[i]=i;
	int ans=0;
	while(k--){
		int op,x,y;
		cin>>op>>x>>y;
		if(x<1 || x>n || y<1 || y>n) {
			ans++;
			continue;
		}
		if(op==1){
			int fx=find(x), fy=find(y);
			if(fx==fy){
				if((dis[y]-dis[x]+3)%3!=0) ans++;
			}
			else{
				dis[fx]=(dis[y]-dis[x]+3)%3;
				fa[fx]=fy;
			}
		}
		else{
			int fx=find(x), fy=find(y);
			if(fx==fy){
				if((dis[y]-dis[x]+3)%3!=1) ans++;
			}
			else{
				dis[fx]=(dis[y]-dis[x]-1+3)%3;
				fa[fx]=fy;
			}
		}
	}
	cout<<ans<<endl;
	return 0;
}
\end{lstlisting}

\subsection{AC自动机}

\begin{lstlisting}[language={[ANSI]C},numbers=left]

struct AC{
	int tr[N][27],num[N],fail[N];	//tr: 字典树数组、num: 记录模式串数量、失配跳转指针
	int tot;	// 点编号
	void insert(string s){	// 插入模式串,构建字典树
		int p=0,len=s.size();
		for(int i=0;i<len;i++){
			if(!tr[p][s[i]-'a']) tr[p][s[i]-'a']=++tot;
			p=tr[p][s[i]-'a'];
		}
		num[p]++;	// 数量加一
		return ;
	}
	queue<int> q;
	void build(){
		for(int i=0;i<26;i++){
			if(tr[0][i]) q.push(tr[0][i]);
		}
		while(!q.empty()){
			int u=q.front();
			q.pop();
			for(int i=0;i<26;i++){
				if(tr[u][i]){
					fail[tr[u][i]]=tr[fail[u]][i];
					q.push(tr[u][i]);
				}
				else tr[u][i]=tr[fail[u]][i];	// 压缩路径
			}
		}
		return ;
	}
	int query(string s){
		int len=s.size(),u=0,ret=0;
		for(int i=0;i<len;i++){
			u=tr[u][s[i]-'a'];
			for(int j=u;j && num[j]!=-1;j=fail[j]){	
			// 每一个字符都当成失配字符算一遍匹配模式串数量
				ret+=num[j];
				num[j]=-1;	// 一个模式串被计算过了就要标记一下
			}
		}
		return ret;
	}
};
AC tree;
\end{lstlisting}

\subsection{树状数组}

\subsubsection{单点修改,区间查询}

\begin{lstlisting}[language={[ANSI]C},numbers=left]

int lowbit(int x){
	return x&-x;
}
void add(int i) {
	while(i<=n) {
		c[i]++;
		i+=lowbit(i);
	}
}
ll getsum(int i) {
	long long res=0;
	while(i>0) {
		res+=c[i];
		i-=lowbit(i);
	}
	return res;
}
\end{lstlisting}

\subsubsection{树状数组求逆序数}

\begin{lstlisting}[language={[ANSI]C},numbers=left]

#include <bits/stdc++.h>
using namespace std;
const int N=500001;
typedef long long ll;
int c[N];
struct Node {
	int v,id;
	bool operator < (const Node &b) const {
		return v<b.v; //从小到大排序
	}
} arr[N];
int n;
int lowbit(int x){
	return x&-x;
}
void add(int i) {
	while(i<=n) {
		c[i]++;
		i+=lowbit(i);
	}
}
ll getsum(int i) {
	long long res=0;
	while(i>0) {
		res+=c[i];
		i-=lowbit(i);
	}
	return res;
}

int main() {
	while(1) {
		cin>>n;
		if(n==0) break;
		int a;
		memset(arr,0,sizeof arr);
		memset(c,0,sizeof c);
		for(int i=1; i<=n; i++) {
			scanf("%d",&a);
			arr[i].id=i;
			arr[i].v=a;
		}
		sort(arr+1,arr+1+n);
		ll ans=0;
		for(int i=1; i<=n; i++) {
			add(arr[i].id);  //离散化结果—— 下标等效于数值
			ans+=i-getsum(arr[i].id); //得到之前有多少个比你大的数(逆序对)
		}
		cout<<ans<<endl;
	}
	return 0;
}
\end{lstlisting}

\subsection{线段树}

\subsubsection{区间维护各种属性}

注意lazy标记的初始化
\begin{lstlisting}[language={[ANSI]C},numbers=left]

#define ls u<<1
#define rs u<<1|1
struct node{
	int l,r;
	int sum;
}tr[N<<2];
int lazy[N<<2];
void pushup(int u){
	tr[u].sum=tr[ls].sum+tr[rs].sum;
}
void pushdown(int u){
	if(!lazy[u]) return ;
	tr[ls].sum+=(tr[ls].r-tr[ls].l+1)*lazy[u];
	tr[rs].sum+=(tr[rs].r-tr[rs].l+1)*lazy[u];
	lazy[ls]+=lazy[u];
	lazy[rs]+=lazy[u];
	lazy[u]=0;	//加法初始化为0 
//	lazy[u]=1;	//乘法法初始化为1
}
void build(int u,int l,int r){
	lazy[u]=0;  //初始化标记 
	if(l==r) tr[u]={l,r,0};
	else{
		tr[u]={l,r};
		int mid=(l+r)>>1;
		build(ls,l,mid);
		build(rs,mid+1,r);
		pushup(u); 
	}
}
void update(int u,int l,int r,int c){
	if(tr[u].l>=l && tr[u].r<=r){
		tr[u].sum+=(tr[u].r-tr[u].l+1)*c;  //区间值加上c 
		lazy[u]+=c;  //更新加法标记 
	}
	else{
		pushdown(u);
		int mid=(tr[u].l+tr[u].r)>>1;
		if(l<=mid) update(ls,l,r,c);
		if(r>mid) update(rs,l,r,c);
		pushup(u);
	}
}
int query(int u,int l,int r){
	if(tr[u].l>=l && tr[u].r<=r) return tr[u].sum;
	else{
		pushdown(u);
		int mid=(tr[u].r+tr[u].l)>>1;
		int res=0;
		if(l<=mid) res=res+query(ls,l,r);
		if(r>mid) res=res+query(rs,l,r);
		return res;
	}
}
\end{lstlisting}

\subsubsection{区间维护最大连续子段和}

\begin{lstlisting}[language={[ANSI]C},numbers=left]

#include <bits/stdc++.h>
#define endl '\n'
#define ls u<<1
#define rs u<<1|1
#define ios ios::sync_with_stdio(0);cin.tie(0);cout.tie(0)
using namespace std;
const int N=500100;
struct node{
	int l,r;
	int lmx,rmx;
	int sum,ms;
}tr[N<<2];
int arr[N];
void pushup(int u){
	tr[u].sum=tr[ls].sum+tr[rs].sum;
	tr[u].lmx=max(tr[ls].lmx,tr[ls].sum+tr[rs].lmx);
	tr[u].rmx=max(tr[rs].rmx,tr[rs].sum+tr[ls].rmx);
	tr[u].ms=max(tr[ls].ms,tr[rs].ms);
	tr[u].ms=max(tr[ls].rmx+tr[rs].lmx,tr[u].ms);
}
void build(int u,int l,int r){
	if(l==r){
		tr[u]={l,r,arr[l],arr[l],arr[l],arr[l]};
	}
	else{
		int mid=l+r>>1;
		tr[u]={l,r};
		build(u<<1,l,mid);
		build(u<<1|1,mid+1,r);
		pushup(u);
	}
}
void update(int u,int pos,int c){
	if(tr[u].l==tr[u].r && tr[u].l==pos){
		tr[u].ms=c;
		tr[u].lmx=c;
		tr[u].rmx=c;
		tr[u].sum=c;
		return ;
	}
	else{
		int mid=tr[u].l+tr[u].r>>1;
		if(pos<=mid) update(ls,pos,c);
		else update(rs,pos,c);
		pushup(u);
	}
}
node query(int u,int l,int r)
{
	if(tr[u].l>=l&&tr[u].r<=r)return tr[u];
	else {
		int mid=(tr[u].l+tr[u].r)>>1;
		if(r<=mid)return query(u<<1,l,r);
		else if(l>mid)return query(u<<1|1,l,r);
		else
		{
			node a=query(u<<1,l,r),b=query(u<<1|1,l,r);
			node c;
			c.sum=a.sum+b.sum;
			c.lmx=max(a.lmx,a.sum+b.lmx);
			c.rmx=max(b.rmx,a.rmx+b.sum);
			c.ms=max(a.rmx+b.lmx,max(a.ms,b.ms));
			return c;
		}
	}
}
int m,n;
signed main()
{
	ios;
	cin>>n>>m;
	for(int i=1;i<=n;i++) cin>>arr[i];
	build(1,1,n);
	while(m--){
		int k,x,y;
		cin>>k>>x>>y;
		if(k==1){
			if(x>y) swap(x,y);
			node tmp=query(1,x,y);
			cout<<tmp.ms<<endl;
		}
		else{
			update(1,x,y);
		}
	}
	
	return 0;
}
\end{lstlisting}

\subsubsection{区间维护最大公约数}

$gcd(a,b)=gcd(a,b-a)$

$(a,b,c)=((a,b),(b,c))=((a,b-a),(b,c-b))=((a,b),(b-a,c-b))=(a,b-a,c-b)$

通过这个式子可以发现,

$gcd(a,b,c,d...)=gcd(a,b-a,c-b,d-c...)$

这样可以维护一个$b[i]=a[i]-a[i-1]$

区间加$d$时,只有$b[l]和b[r+1]$的位置需要修改,也就变成了单点修改,区间查询问题,需要注意r+1<=n,还有查询$gcd(l~r)$时,$a[l]$是原值,不是$b[i]$,而$sum(a[1]~a[l])$正好是原值,就省去了懒标记操作了
\begin{lstlisting}[language={[ANSI]C},numbers=left]

#include<bits/stdc++.h>
using namespace std;
typedef long long ll;
const int maxn = 5e5 + 23;
struct Node {
	int l, r;
	ll v, d;
} tr[maxn * 4];
ll a[maxn], b[maxn];
void pushup(Node &u, Node &l, Node &r) {
	u.v = l.v + r.v;
	u.d = __gcd(l.d, r.d);
}
void pushup(int u) {
	pushup(tr[u], tr[u << 1], tr[u << 1 | 1]);
}
void build(int u, int l, int r) {
	if(l == r) tr[u] = {l, r, b[l], b[l]};
	else {
		tr[u] = {l, r};
		int mid = l + r >> 1;
		build(u << 1, l, mid);
		build(u << 1 | 1, mid + 1, r);
		pushup(u);
	}
}
ll query(int u, int l, int r) {		//区间gcd 
	if(tr[u].l >= l && tr[u].r <= r) return tr[u].d;
	else {
		int mid = tr[u].l + tr[u].r >> 1;
		if(r <= mid) return query(u << 1, l, r);
		else if(l > mid) return query(u << 1 | 1, l, r);
		else return __gcd(query(u << 1, l, r), query(u << 1 | 1, l, r));
	}
}
ll query2(int u, int l, int r) {	//区间和 
	if(tr[u].l >= l && tr[u].r <= r) return tr[u].v;
	else {
		int mid = tr[u].l + tr[u].r >> 1;
		if(r <= mid) return query2(u << 1, l, r);
		else if(l > mid) return query2(u << 1 | 1, l, r);
		else return query2(u << 1, l, r) + query2(u << 1 | 1, l, r);
	}
}
void modify(int u, int p, ll v) {
	if(tr[u].l == tr[u].r && tr[u].l == p) tr[u].d += v, tr[u].v += v;
	else {
		int mid = tr[u].l + tr[u].r >> 1;
		if(p <= mid) modify(u << 1, p, v);
		else modify(u << 1 | 1, p, v);
		pushup(u);
	}
}
int main() {
	int n, m;
	scanf("%d%d", &n, &m);
	for(int i = 1; i <= n; i++) scanf("%lld", &a[i]), b[i] = a[i] - a[i - 1];
	build(1, 1, n);
	char op[4];
	ll op1 = 0, op2 = 0, op3 = 0;
	while(m--) {
		scanf("%s%lld%lld", op, &op1, &op2);
		if(*op == 'C') {
			scanf("%lld", &op3);
			modify(1, op1, op3);
			if(op2 + 1 <= n) modify(1, op2 + 1, -op3);
		} else {
			ll t = query2(1, 1, op1);
			printf("%lld\n", abs( __gcd(t, query(1, op1 + 1, op2))));
		}
	}
}
\end{lstlisting}

\subsubsection{区间同时维护乘法和加法}

注意lazy标记的初始化
\begin{lstlisting}[language={[ANSI]C},numbers=left]

#include <bits/stdc++.h>
#define ios ios::sync_with_stdio(0); cin.tie(0); cout.tie(0)
using namespace std;
typedef long long ll;
const ll MOD=1e9+7;
const ll SUP=0x800000;
const ll MAXN=1e5+10;
const ll INF=0x3f3f3f3f;
const double eps=1e-4;
struct node{
	ll l,r,val;
}tr[MAXN<<2];
//lazy1加法标记 lazy2乘法标记 
ll lazy1[MAXN<<2],lazy2[MAXN<<2],arr[MAXN];
ll n,m,p;
void pushup(ll u){
	tr[u].val=(tr[u<<1].val+tr[u<<1|1].val)%p;
}
void pushdown(ll u){
	ll l=u<<1,r=u<<1|1;
	// 儿子的值=儿子的值*父亲的乘法标记+儿子区间*父亲加法标记 
	tr[l].val=(tr[l].val*lazy2[u]%p+(tr[l].r-tr[l].l+1)*lazy1[u]%p)%p;
	tr[r].val=(tr[r].val*lazy2[u]%p+(tr[r].r-tr[r].l+1)*lazy1[u]%p)%p;
	// 儿子加法标记更新,儿子加法标记=儿子加法标记*父亲乘法标记+父亲加法标记 
	lazy1[l]=(lazy1[l]*lazy2[u]%p+lazy1[u])%p;
	lazy1[r]=(lazy1[r]*lazy2[u]%p+lazy1[u])%p;
	//  儿子乘法标记=儿子乘法标记*父亲乘法标记
	lazy2[l]=lazy2[l]*lazy2[u]%p;
	lazy2[r]=lazy2[r]*lazy2[u]%p;
	
	lazy1[u]=0;
	lazy2[u]=1;
}
void build(ll u,ll l,ll r){
	lazy1[u]=0;  //初始化标记 
	lazy2[u]=1;
	if(l==r) tr[u]={l,r,arr[l]};
	else{
		tr[u]={l,r};
		ll mid=(l+r)>>1;
		build(u<<1,l,mid);
		build(u<<1|1,mid+1,r);
		pushup(u); 
	}
}
void add(ll u,ll l,ll r,ll c){
	if(tr[u].l>=l && tr[u].r<=r){
		tr[u].val=(tr[u].val+(tr[u].r-tr[u].l+1)*c%p)%p;  //区间值加上c 
		lazy1[u]=(c+lazy1[u])%p;  //更新加法标记 
	}
	else{
		pushdown(u);
		ll mid=(tr[u].l+tr[u].r)>>1;
		if(l<=mid) add(u<<1,l,r,c);
		if(r>mid) add(u<<1|1,l,r,c);
		pushup(u);
	}
}
void mul(ll u,ll l,ll r,ll c){
	if(tr[u].l>=l && tr[u].r<=r){
		tr[u].val=tr[u].val*c%p;  //区间的值乘上c 
		lazy1[u]=(lazy1[u]*c)%p;  //每次更新乘法标记时要顺带着把加法标记也更新了,目的是确定优先级 
		lazy2[u]=(lazy2[u]*c)%p;  //更新乘法标记 
	}
	else{
		pushdown(u);
		ll mid=(tr[u].l+tr[u].r)>>1;
		if(l<=mid) mul(u<<1,l,r,c);
		if(r>mid) mul(u<<1|1,l,r,c);
		pushup(u);
	}
}
ll query(ll u,ll l,ll r){
	if(tr[u].l>=l && tr[u].r<=r) return tr[u].val;
	else{
		pushdown(u);
		ll mid=(tr[u].r+tr[u].l)>>1,ret=0;
		if(l<=mid) ret=(ret+query(u<<1,l,r))%p;
		if(r>mid) ret=(ret+query(u<<1|1,l,r))%p;
		return ret;
	}
}
int main()
{
	ios;
	cin>>n>>m>>p;
	for(ll i=1;i<=n;i++) cin>>arr[i];
	build(1,1,n); 
	while(m--){
		ll op,x,y,k;
		cin>>op;
		if(op==1){
			cin>>x>>y>>k;
			mul(1,x,y,k);
		}
		else if(op==2){
			cin>>x>>y>>k;
			add(1,x,y,k);
		}
		else{
			cin>>x>>y;
			cout<<query(1,x,y)<<'\n';
		}
	}
	return 0;
 }
\end{lstlisting}

\subsection{维护n棵线段树}
例题:

初始有n个1,接下来q次操作

1. MULTIPLY l r x(x<=10):给[l,r]区间上的数字乘上x

2. MAX l r:求出区间[l,r]内哪一个质因子数量最多,输出数量

因为x<=10,所以最后[1,n]区间内的数字最多就4种质因子,2,3,5,7,给每一个质因子建议一颗线段树,区间乘上x,就对x进行质因子分解,去每一颗质因子的线段树上进行区间加的修改,询问的时候输出四种质因子区间内最大数是多少即可。
\begin{lstlisting}[language={[ANSI]C},numbers=left]
#include <bits/stdc++.h>
#define ls u<<1
#define rs u<<1|1
#define ios ios::sync_with_stdio(0);cin.tie(0);cout.tie(0)
#define endl '\n'
using namespace std;
const int N=1e5+100;
int n,q;
struct node{
	int l,r;
	int sum,mx;
}tr[10][N<<2];
int lazy[10][N<<2];
void pushup(int id,int u){
	tr[id][u].sum=tr[id][ls].sum+tr[id][rs].sum;
	tr[id][u].mx=max(tr[id][ls].mx,tr[id][rs].mx);
}
void pushdown(int id,int u){
	if(lazy[id][u]){
		tr[id][ls].sum+=(tr[id][ls].r-tr[id][ls].l+1)*lazy[id][u];
		tr[id][rs].sum+=(tr[id][rs].r-tr[id][rs].l+1)*lazy[id][u];
		tr[id][ls].mx+=lazy[id][u];
		tr[id][rs].mx+=lazy[id][u];
		lazy[id][ls]+=lazy[id][u];
		lazy[id][rs]+=lazy[id][u];
		lazy[id][u]=0;
	}
}
void build(int id,int u,int l,int r){
	if(l==r) tr[id][u]={l,r,0,0};
	else{
		int mid=l+r>>1;
		tr[id][u]={l,r};
		build(id,ls,l,mid);
		build(id,rs,mid+1,r);
	}
}
void update(int id,int u,int l,int r,int c){
	if(tr[id][u].l>=l && tr[id][u].r<=r){
//		cout<<tr[id][u].l<<" "<<tr[id][u].r<<" "<<l<<" "<<r<<endl;
		tr[id][u].sum+=c*(tr[id][u].r-tr[id][u].l+1);
		tr[id][u].mx+=c;
		lazy[id][u]+=c;
	}
	else{
		pushdown(id,u);
		int mid=tr[id][u].l+tr[id][u].r>>1;
		if(l<=mid) update(id,ls,l,r,c);
		if(r>mid) update(id,rs,l,r,c);
		pushup(id,u);
	}
}
int query(int id,int u,int l,int r){
	if(tr[id][u].l>=l && tr[id][u].r<=r) return tr[id][u].mx;
	else{
		pushdown(id,u);
		int mid=tr[id][u].l+tr[id][u].r>>1;
		int res=0;
		if(l<=mid) res=max(res,query(id,ls,l,r));
		if(r>mid) res=max(res,query(id,rs,l,r));
		return res;
	}
}
int mp[10];
int main()
{
	ios;
	cin>>n>>q;
	mp[2]=1; mp[3]=2; mp[5]=3; mp[7]=4;
	for(int i=1;i<=4;i++) build(i,1,1,n);
	while(q--){
		string s;
		cin>>s;
		if(s[1]=='U'){
			int l,r,x;
			cin>>l>>r>>x;
			for(int i=2;i<=sqrt(x);i++){
				if(x%i==0){
					int cnt=0;
					while(x%i==0){
						cnt++;
						x/=i;
					}
					update(mp[i],1,l,r,cnt);
				}
			}
			if(x>1) update(mp[x],1,l,r,1);
		}
		else{
			int l,r;
			cin>>l>>r;
			int ans=0;
			for(int i=1;i<=4;i++){
				int tmp=query(i,1,l,r);
				ans=max(ans,tmp);
			}
			cout<<"ANSWER "<<ans<<endl;
		}
	}
	return 0;
}
\end{lstlisting}

\subsection{主席树(可持久化权值线段树)}

\subsubsection{求区间第K大、第K小问题}

\begin{lstlisting}[language={[ANSI]C},numbers=left]

#include <bits/stdc++.h>
#define ios ios::sync_with_stdio(0);cin.tie(0);cout.tie(0)
using namespace std;
const int N=1e6+100;
struct node {
	int ls,rs,sum;
} tr[N*40];
int rt[N*40];
int arr[N],brr[N];
int n,m,tot;
int build(int l,int r) {	//新建一颗空树
	int root=++tot;
	if(l==r) return root;
	int mid=l+r>>1;
	tr[root].ls=build(l,mid);
	tr[root].rs=build(mid+1,r);
	return root;
}
int insert(int pre,int l,int r,int pos) {	//在上一个版本的基础上新建一个版本并插入pos值
	int root=++tot;
	tr[root]=tr[pre];
	tr[root].sum++;
	if(l==r)  return root;
	int mid=(l+r)>>1;
	if(pos<=mid) tr[root].ls=insert(tr[pre].ls,l,mid,pos);
	else tr[root].rs=insert(tr[pre].rs,mid+1,r,pos);
	return root;
}
int query_k_min(int u,int v,int l,int r,int k) {	//查询第k小 
	if(l==r) return l;
	int cha=tr[tr[v].ls].sum-tr[tr[u].ls].sum;
	int mid=l+r>>1;
	if(cha>=k) return query_k_min(tr[u].ls,tr[v].ls,l,mid,k);
	else return query_k_min(tr[u].rs,tr[v].rs,mid+1,r,k-cha);
}
int query_k_max(int u,int v,int l,int r,int k) {	//查询第k大 
	if(l==r) return l;
	int cha=tr[tr[v].rs].sum-tr[tr[u].rs].sum;
	int mid=l+r>>1;
	if(cha>=k) return query_k_max(tr[u].rs,tr[v].rs,l,mid,k);
	else return query_k_max(tr[u].ls,tr[v].ls,mid+1,r,k-cha);
}
int main() {
	ios;
	cin>>n>>m;
	for(int i=1; i<=n; i++) {
		cin>>arr[i];
		brr[i]=arr[i];
	}
	sort(arr+1,arr+1+n);	//离散化,brr存离散值,arr原值
	for(int i=1; i<=n; i++) brr[i]=lower_bound(arr+1,arr+1+n,brr[i])-arr;
	
	rt[0]=build(1,n);
	for(int i=1; i<=n; i++) rt[i]=insert(rt[i-1],1,n,brr[i]);
	
	while(m--) {
		int l,r,k;
		cin>>l>>r>>k;
		cout<<arr[query_k_min(rt[l-1],rt[r],1,n,k)]<<'\n';
	}
	return 0;
}
\end{lstlisting}

\subsubsection{维护区间内存在多少个不同的数字(单纯的权值线段树做不到)}


\begin{lstlisting}[language={[ANSI]C},numbers=left]

int query_sum(int u,int v,int l,int r,int ql,int qr){	//查询区间和
	if(l>=ql && r<=qr) return tr[v].sum-tr[u].sum;
	else{
		int mid=l+r>>1;
		int res=0;
		if(ql<=mid) res+=query_sum(tr[u].ls,tr[v].ls,l,mid,ql,qr);
		if(qr>mid) res+=query_sum(tr[u].rs,tr[v].rs,mid+1,r,ql,qr);
		return res;
	}
}
\end{lstlisting}

\subsection{字典树}

\section{图论}
\subsection{tarjan}
缩点经常和拓扑dp一起使用,缩点的目的就是把图中所有可以互相到达的点看成一个点(强连通分量),然后开一个数组维护强连通分量内部的信息,缩点后图就变成了DAG有向无环图,跑拓扑的原因是可以降低时复,bfs会跑重复的点,例如一条链,最开始的地方有一个分叉,之后又合上了,但是后面的部分非常长,这样小分叉多点,时复就炸了
\begin{lstlisting}[language={[ANSI]C},numbers=left]
#include<bits/stdc++.h>
#define endl '\n'
#define ios ios::sync_with_stdio(false),cin.tie(0),cout.tie(0)
using namespace std;
const int N=1e5+100;

//链式前向星存图 
struct Edge{
	int to,next;
}e[1000100];
int h[N],tot;
void add(int u,int v){
	e[tot]={v,h[u]};
	h[u]=tot++;
}

/*
dfn: 时间戳
siz: 联通块内部点数量
idx: 点所属联通块编号
instk: 是否在栈中 
*/ 
int dfn[N],low[N],siz[N],idx[N];
bool instk[N];
int tim,cnt; 	//时间戳,联通块数量 
stack<int> stk;
void tarjan(int x){		//缩点 
	dfn[x]=low[x]=++tim;
	stk.push(x);
	instk[x]=1;
	for(int i=h[x];~i;i=e[i].next){
		int v=e[i].to;
		if(!dfn[v]){
			tarjan(v);
			low[x]=min(low[v],low[x]);
		}
		else if(instk[v]) low[x]=min(low[x],dfn[v]);
	}
	if(low[x]==dfn[x]){
		cnt++;
		while(stk.top()!=x){
			siz[cnt]++;
			idx[stk.top()]=cnt;
			instk[stk.top()]=0;
			stk.pop();
		}
		siz[cnt]++;
		idx[stk.top()]=cnt;
		instk[stk.top()]=0;
		stk.pop();
	}
}
/*
G是缩点后的图
in: 入度 
mi: 联通块内部点权最小 
mx: 联通块内部点权最大 
*/ 
vector<int> G[N];
int in[N],mi[N],mx[N];
void tupo(){
	queue<int> q;
	//从这里开始是把1前面的点跑拓扑给释放掉,因为起点是1,不释放的话会把起点之前的点也算进去 
	for(int i=1;i<=cnt;i++){
		if(!in[i] && i!=idx[1]) q.push(i);
	}
	while(!q.empty()){
		int u=q.front();
		q.pop();
		for(auto v:G[u]){
			if(v!=idx[1]) in[v]--;
			if(!in[v]) q.push(v);
		}
	}
	//到这里结束 
	q.push(idx[1]);
	while(!q.empty()){
		int u=q.front();
		q.pop();
		for(auto v:G[u]){
			/*
			mi[v]=min(mi[v],mi[u]);
			ans[v]=max(ans[v],mx[v]-mi[v]);
			ans[v]=max(ans[v],ans[u]);
			*/
			in[v]--;
			if(!in[v]) q.push(v);
		}
	}
}

int n,m;
int num[N];
int main() {
	ios;
	memset(h,-1,sizeof h);	//记得初始化 
	cin>>n>>m;
	for(int i=1;i<=n;i++) cin>>num[i];	
	while(m--){
		int u,v,op;
		cin>>u>>v>>op;
		add(u,v);
		if(op==2) add(v,u);
	}
	memset(mi,0x3f,sizeof mi);
	//图可能不连通,所以需要遍历所有点 
	for(int i=1;i<=n;i++){	
		if(!dfn[i]) tarjan(i);
	}
	//每一个点去更新他所在联通块的最值 
	for(int i=1;i<=n;i++){
		mi[idx[i]]=min(mi[idx[i]],num[i]);
		mx[idx[i]]=max(mx[idx[i]],num[i]);
	}
	//建新图 
	for(int i=1;i<=n;i++){
		for(int j=h[i];~j;j=e[j].next){
			int v=e[j].to;
			if(idx[v]!=idx[i]) G[idx[i]].push_back(idx[v]);
		}
	}
	//新图入度 
	for(int i=1;i<=cnt;i++){
		for(auto v:G[i]){
			in[v]++;
		}
	}
	//拓扑 
	tupo();
	cout<<ans[idx[n]]<<endl;
	return 0; 
}
/*
6 6
2 6 2 1 8 9
1 4 1
4 5 1
5 1 1
1 2 1
2 3 1
2 6 1
*/
\end{lstlisting}
\subsection{floyed求最小环}
\begin{lstlisting}[language={[ANSI]C},numbers=left]
for(int k=1;k<=n;k++){
    for(int i=1;i<k;i++){	//这段代码执行时k还没有更新其他点的最短距离
        for(int j=i+1;j<k;j++){
                ans=min(ans,dis[i][j]+G[j][k]+G[k][i]);
        }
    }
    for(int i=1;i<=n;i++){
        for(int j=1;j<=n;j++){
            dis[i][j]=min(dis[i][j],dis[i][k]+dis[k][j]);
        }
    }
}
\end{lstlisting}
\section{数论}

\section{杂类}
\subsection{int128}
\begin{lstlisting}[language={[ANSI]C},numbers=left]
void read(int128 &x) {
	x = 0;
	int f = 1;
	char ch;
	if((ch = getchar()) == '-') f = -f;
    else x = x*10 + ch-'0';
    while((ch = getchar()) >= '0' && ch <= '9')
        x = x*10 + ch-'0';
	x *= f;
}
void print(int128 x) { //输出
	if(x < 0) {
		x = -x;
		putchar('-');
	}
    if(x > 9) print(x/10);
    putchar(x%10 + '0');
}
\end{lstlisting}

\subsection{O(3)优化}

\begin{lstlisting}[language={[ANSI]C},numbers=left]
#pragma GCC optimize(3,"Ofast","inline")
\end{lstlisting}

\subsection{模拟退火}
\begin{lstlisting}[language={[ANSI]C},numbers=left]
//随机数
srand(time(NULL));
int rand_INT(int l,int r) {	//产生[l,r]的的一个随机数
	return rand()%(r-l+1)+l;
}
double rand_DOUBLE(double l,double r){	//产生[l,r]的的一个随机数
	return (double)rand()/RAND_MAX*(r-l)+l;
}
//以Acwing进阶课题目为例
#include <bits/stdc++.h>
#define ios ios::sync_with_stdio(0);cin.tie(0);cout.tie(0)
#define endl '\n'
using namespace std;
const int N=110;
struct Point{
	double x,y;
}q[N];
int n;
double ans=1e8;
double rand(double l,double r){	//产生[l,r]的的一个随机数
	return (double)rand()/RAND_MAX*(r-l)+l;
}
double getdis(Point x){
	double res=0;
	for(int i=1;i<=n;i++){
		res+=sqrt(pow(q[i].x-x.x,2)+pow(q[i].y-x.y,2));
	}
	ans=min(res,ans);
	return res;
}
void simulate_anneal(){	//退火函数
	Point now={rand(0,10000),rand(0,10000)};
	for(double t=1e4;t>1e-4;t*=0.999){	//t是温度,
		Point np={rand(0,10000),rand(0,10000)};	//np是在[x-t,x+t]范围内产生的新点
		double dt=getdis(np)-getdis(now);	//dt是新点函数值和旧点函数值的差值
        //这里需要注意这道题取得是最小值,所以加了负号,如果求最大值就要把负号去掉!!
		if(exp(-dt/t)>rand(0,1)) now=np;	//如果新点好就一定跳过去,否则以一定概率跳过去
	}
}
int main()
{
    ios;
    cin>>n;
    for(int i=1;i<=n;i++) cin>>q[i].x>>q[i].y;
    for(int i=1;i<=100;i++) simulate_anneal();
    cout<<fixed<<setprecision(0)<<ans<<endl;
    return 0;
}
\end{lstlisting}
\subsection{手写Hash(int)}
\begin{lstlisting}[language={[ANSI]C},numbers=left]
struct MAP{
    int tot;
    int nxt[M],las[P],to[M],w[M];
    MAP(){
        tot=0;
        for(int i=0;i<=P-1;++i)
            las[i]=0;
    }
    void insert(int y){
        int x=y%P;
        for(int e=las[x];e;e=nxt[e]){
            if(to[e]==y){
                ++w[e];
                return;
            }
        }
        nxt[++tot]=las[x];
        las[x]=tot;
        to[tot]=y;
        w[tot]=1;
    }
    int find(int y){
        int x=y%P;
        for(int e=las[x];e;e=nxt[e]){
            if(to[e]==y)
                return w[e];
        }
        return 0;
    }
} mp1;
\end{lstlisting}

section{字符串}
subsection{字符串哈希}
\begin{lstlisting}[language={[ANSI]C},numbers=left]
typedef unsigned long long ull;
ull ha[N];
ull po[N];
ull bas=1331;
void init_hash(string &s){
	int len=s.size();
	s=" "+s;
	po[0]=1; ha[0]=0;
	for(int i=1;i<=len;i++){
		ha[i]=ha[i-1]*bas+s[i]-'0';
		po[i]=po[i-1]*bas;
	}
}
ull get_hash(int l,int r){
	return ha[r]-ha[l-1]*po[r-l+1];
}
\end{lstlisting}

subsection{KMP}
\begin{lstlisting}[language={[ANSI]C},numbers=left]
int ne[N];	//next数组: i位置之前的最长公共前后缀
void get_next(string &s){
	int len=s.size();
	int i=0, j=-1;
	ne[0]=-1;
	while(i<len){
		if(j==-1 || s[i]==s[j]){
			++i; ++j;
			ne[i]=j;
		}
		else j=ne[j];
	}
	return ;
}
int kmp(string &s,string &t){
	get_next(t);
	int len1=s.size(),len2=t.size();
	int i=0, j=0, cnt=0;
	while(i<len1){
		if(j==-1 || s[i]==t[j]){
			++i; ++j;
		}
		else j=ne[j];
		if(j==len2) cnt++;
	}
	return cnt;
}
//nextval优化
void get_nextval(string s){
	int i=0, j=-1;
	nextval[0]=-1;
	int len=s.size();
	while(i<len){
		if(j==-1 || s[i]==s[j]){
			++i; ++j;
            // 和求解next数组唯一不一样的地方
			if(s[i]!=s[j]) nextval[i]=j;  
			else nextval[i]=nextval[j]; //当前位置和next[i]的字符相同时,则不需要回朔到next[i]位置,因为这个位置的字符一定会失配,所以让nextval[i]直接指向nextval[next[i]],当没有跳步回朔操作时next和nextval值是一样的
		}
		else j=nextval[j];
	}
}
\end{lstlisting}

subsection{扩展KMP}

\begin{lstlisting}[language={[ANSI]C},numbers=left]
void getnext(string T){
	int len=T.size();
	nex[0]=len;
	int p=0;
	while(p+1<len && T[p]==T[p+1]) p++;  // 这里注意把边界写在前面
	nex[1]=p;
	int po=1;
	for(int i=2;i<len;i++){
		if(i+nex[i-po]<po+nex[po]) nex[i]=nex[i-po];  // 第一种情况,直接得到答案
		else{
			int j=po+nex[po]-i;  
			if(j<0) j=0;  // 超出已匹配的字符串长度,需要重新匹配
			while(i+j<len && T[i+j]==T[j]) j++;
			nex[i]=j;
			po=i;  // 长度超出,更新起始位置
		}
	}
}
void extmp(string S,string T){
	int len1=S.size(), len2=T.size();
	getnext(T);
	int p=0;
	while(p<len1 && p<len2 && S[p]==T[p]) p++;  // 边界写到前面
	ext[0]=p;
	int po=0;
	for(int i=1;i<len1;i++){
		if(i+nex[i-po]<po+ext[po]) ext[i]=nex[i-po];
		else{
			int j=po+ext[po]-i;
			if(j<0) j=0;
			while(i+j<len1 && j<len2 && T[j]==S[i+j]) j++;
			ext[i]=j;
			po=i;
		}
	}
}
\end{lstlisting}

\subsection{马拉车算法}
\begin{lstlisting}[language={[ANSI]C},numbers=left]
int p[N];
string s;
int manacher(string s){
	string t="";
	t+='*';
	for(int i=0;i<(int)s.size();i++){
		t+=s[i];
		t+='*';
	}
	int ans=0;
	int pos=0;int maxxright=0;
	for(int i=0;i<(int)t.length();i++){
		p[i]=maxxright>i?min(p[2*pos-i],maxxright-i):1;//关键
		while(i-p[i]>=0&&i+p[i]<(int)t.length()&&t[i-p[i]]==t[i+p[i]]) p[i]++;
		if(i+p[i]-1>maxxright){
			maxxright=i+p[i]-1;
			pos=i;
		}
		ans=max(ans,p[i]);
	}
	return ans-1;
}
\end{lstlisting}

\end{document}
